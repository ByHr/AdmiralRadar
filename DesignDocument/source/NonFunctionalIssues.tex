\subsubsection*{Issue: \textit{What kind of database will we use?}}

  \begin{itemize}
    \item MongoDB
    \item \textbf{SQL}
  \end{itemize}

The game is structured in such a way where the stored data would be easily interconnected through relational entities, making SQL (or any SQL option) the simplest to implement.

\subsubsection*{Issue: \textit{How will a team play with fewer than four members?}}

  \begin{itemize}
    \item An AI would take control of the slots lacking players.
    \item \textbf{Allow a user to occupy multiple spots.}
  \end{itemize}

This is a very simple fix and potentially the most fun for players. The challenge of having fewer people control a ship creates real tension and greater reward should this team pull off a victory against the odds. Of course, we will look into developing an AI at some point in the future which would allow the players to decide either option as they wish.

\subsubsection*{Issue: \textit{How will users be delayed in gameplay during a spacewalking action? (equivalent to a "surfacing" in original game)}}

  \begin{itemize}
    \item \textbf{One at a time, each user will have to trace a shape using their mouse in order to proceed.}
    \item The users would have to each move at the same time with the arrow keys.
  \end{itemize}

Having to go one at a time brings back a bit of that board game feel where players share the same physical space and creates tension with the entire team watching one player move across the board.

\subsubsection*{Issue: \textit{How will the clients handle a server disconnection?}}

  \begin{itemize}
    \item \textbf{Pause the game for up to around 30 seconds and, if a connection isn't restored, return the user to the lobby.}
    \item Once disconnection occurs, push any actions along with time stamps held by the clients on a temporary queue to then be sent to the server once connection is restored and the server will sort each action in the order of the time performed and execute it in that order. 
  \end{itemize}

Pausing the game and having players wait will be the simplest to implement given the time restraints, and even we were able to implement the other option competently, players might find a way to leverage server timing to get certain actions performed before others, so it also acts as a preventative cheating measure.

\subsubsection*{Issue: \textit{How will the server handle a database disconnection?}}

  \begin{itemize}
    \item \textbf{Send a message to the player letting them know what happened, then allow the game to complete and cache results until connection is re-established. No further login would be allowed during that time.}
    \item Have the player on a loading screen at startup until connection is reestablished.
  \end{itemize}

Having the player know what is wrong with the internals of the game as it happens will dispel any confusion with the issue. The player won't have to troubleshoot the error themselves.

\subsubsection*{Issue: \textit{On what hardware will the server be implemented?}}

  \begin{itemize}
    \item It will be implemented on a personal Linux desktop.
    \item \textbf{It will be implemented on Linux servers operated by team members.}
  \end{itemize}

Linux servers are the most versatile and are widely used throughout the world because of that fact. We are also most experienced with implementing Linux servers so it would save a lot of development time.
