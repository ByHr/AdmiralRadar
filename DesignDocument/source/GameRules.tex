The game of \textit{Admiral Radar} is based on the rules of the board game \textit{Captain Sonar}, which are described here. The game is normally played by two teams of four people, each of which represent the crew on a submarine during wartime. The overall objective of the game is to sink the opposing submarine by removing all four of its health points. Each member of the crew has a unique role: Captain, First Officer, Radio Officer, and Engineer. Each of these roles performs different functions, all of which are necessary for the submarine to successfully navigate, find, and damage the opposing team's submarine. In the physical game, each role has a different board design in order to accomplish these tasks. The teams are separated by a large physical barrier to prevent one team from seeing the other team's boards.

The game is played in turns, with each turn consisting of a series of moves by the Captain, First Officer, and Engineer. Most significantly, the turns of the two submarines are not necessary coordinated. If the game is played in "turn-based" mode, one submarine completes its turn and then the other submarine does the same, trading back and forth until the end of the game. If the game is played in "real time" mode, however, one submarine may begin their next turn as soon as it completes the previous one: the turns of the teams are not synchronized. In both cases, the actions of the Radio Operator are linked to the actions performed by both submarines, although their play will inform the choices made by their team's Captain during their turn. 

The Captain's game board has a dot-grid diagram of the game map. This grid has certain points blocked off, representing islands. At the beginning of the game, the Captains will select a starting point for their submarine and, using an erasable marker, note it on their board. Every turn is begun by the Captain commanding a direction: North, South, East, or West, for the submarine to travel. They must vocally announce this for all players to hear. After the First and Engineering Officers have completed their tasks, the Captain will mark the direction taken on their board by drawing a line to the dot in the direction on the map which they selected, indicating that the submarine has moved to this new position on the map. They may not direct the submarine in any direction that would contact an island or the boundaries of the map. Additionally, they may not move the submarine in such a way that it would cross its previous path.

\newsamimage[0.6]{Game Map}{map}

After the Captain has ordered a direction, the First and Engineering Officers complete their responsibilities. The board of the First Officer has a series of six gauges representing the ship's six special systems: Torpedoes, Mines, Drones, Sonar, Silence, and Scenario.  Every turn, they distribute one unit of power to a system of their choice, and, once complete, announce "done" to the Captain. If they have completely filled a gauge, they must also vocally announce that the system corresponding to that gauge is ready.

\newsamimage[0.6]{First Officer's Game Board}{firstofficer}

At the same time, the Engineering Officer is performing their own task. With every move, they must select a certain ship system to disable for maintenance. Which components can be disabled are dependent on the direction that the Captain has announced. Depending on the Engineer's choice, a component of the sensory, tactical, propulsion, or axillary systems will be disabled, making those systems unavailable for the Captain's use. If, however, the Engineer disables all members of a specific set (called a circuit) of components, those components will be repaired, and the systems they belong to will again be available for the Captain's use. Some components cannot be repaired in this way, as they are not part of one of the circuits on the Engineer's board. Of special significance are the reactor components. While the Engineer choosing to take them offline does not diable a submarine system, if all reactor components are disabled, the submarine looses one point of health and the Engineer re-enables all systems. Because the reactor components are not part of any circuit, they cannot be repaired except by the Surfacing mechanic (discussed later). After marking a component as disabled, or, if a circuit is complete, marking the component enabled again, the Engineer, like the First Officer, announces "done." Once both the Engineer and First Officer have announced that they are done, the turn is complete. 

\newsamimage[0.6]{Engineering Officer's Game Board}{engineer}

The final role on the submarine is that of the Radio Officer, whose job is to determine the location of the enemy submarine. Like the Captain, their board has a dot-grid map of the play area; however their job is to draw the path of the opposing submarine, which they learn by listening to the opposing Captain's directional commands. While this means that they know the path of the enemy submarine, they do not know the starting location, which they must determine from available clues. As the game goes on and the submarine's path gets longer, fewer and fewer coordinates become viable guesses for the actual starting location of the enemy submarine. The Radio Officer draws the path of their opponent's vessel on a transparent plastic sheet, which they may move overtop the map to determine its present location. When the Radio Officer has a good guess of where the enemy is, they notify their Captain, who can use that information to attack their opponent.

\newsamimage[0.6]{Radio Officer's Station}{radioofficer}

In addition to movement, the Captain has, at their disposal, seven special actions which they may order the submarine to perform. The first six actions must be charged by the First Officer before being available for the Captain's use, while the seventh, "surfacing," may be ordered at any time. Firing a torpedo is one of the two tactical actions of the game. When the Captain fires a Torpedo, they select a grid location up to four cardinal steps from the location of their submarine and report it to the opposing Captain. The opposing Captain compares the announced location to the location of their submarine: if the announced coordinate is the same as the one which the submarine currently inhabits, the submarine looses two health points. If the selected coordinate is adjacent to the submarine in any direction, it looses one health point. Otherwise, the enemy submarine looses no health. The opposing Captain will announce how much health their submarine has lost from the Torpedo. The other tactical system, Mines, work slightly differently. At any point, the Captain announces that they are laying a Mine, and indicates to a Radio Officer a coordinate one unit away from the current location of their submarine in any direction, which the Radio Officer records on their map (not their overlay). At any point after, the Captain may then remotely detonate that Mine, which will cause damage in the same pattern as if a Torpedo was targeted at the location of the Mine. 

\newsamimage[0.6]{Torpedo Firing Diagram}{torpedo}

The next two systems are the sensory systems: Drones and Sonar. Both of these provide information about the location of the enemy submarine to the Radio Officer. When the Drone is used, the Captain selects one of the major grid squares (called sectors) on the map to test for the presence of the enemy submarine, announcing the number of that sector to the opposing Captain. The opposing Captain then answers, truthfully, if their submarine is located in that sector or not. The Radio Officer then records this information on their map, as it allows them to eliminate a number of potential locations at which the enemy submarine could possibly be located. The Sonar action provides a similar general function, although the exact information which the enemy must provide is different. If a ship uses its Sonar, its opponent must announce two pieces of information about their location from three possible choices: the row in which they are located in, the column they occupy, or the number of their current sector. Although two pieces of information are provided, only one of these is truthful: the other is deliberately false. Like when executing the Drone action, the Radio Officer will mark down the information they have gained about the enemy's current position on their board.

The fifth action which the submarine may perform is to Silence itself. If the Captain orders the submarine to Silence, it instantly jumps up to four coordinates away, in a straight line. No information, besides that the Silence action was performed, is reported to the opposing team. Like a standard movement, a Silenced movement may not intersect with an island, map edge, or the submarine's previous path. Unlike the other four discussed actions, performing a Silence does not pause the game (which is done when a Torpedo, Mine, Drone, or Sonar action is used) until it has been completed: both submarines continue taking their turns. The Scenario action, the final one available to the submarine, does nothing in the basic version of the game: it can perform special actions depending on specialized rules for alternate game variants, which are beyond the scope of this description.

\newsamimage[0.6]{Silenced Movement}{silence}

Finally, at any time the Captain may order the submarine to Surface. They must also inform the enemy team which sector the submarine occupies. Once a Surface is performed, the Engineer re-enables all submarine systems which have been disabled. The Captain may clear the previous route from their board, allowing the submarine more room to maneuver, as it does not have to avoid the path it followed prior to surfacing. If the game is being played in turn-based mode, the enemy submarine gets three turns. If the game is being played in real-time mode, the enemy continues to perform actions while the surfaced submarine performs a task designed to waste time (tracing the outline of certain shapes on the Engineer's player board). During the time the submarine is surfaced, its Radio Operator may not record the enemy submarine's movements: they must remember and record the movements once the submarine has submerged again. At this time gameplay resumes as normal.

As soon as one submarine has lost four units of health, either from reactor damage or enemy weapons, the submarine is sunk and looses the game. The surviving submarine crew is declared the winning team.







