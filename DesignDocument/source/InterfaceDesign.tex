\samimage[GUI Model]{InterfaceModel}

\textit{Admiral Radar}'s desktop client software will provide the user with a GUI for interactive gameplay. The GUI will connect to the backend classes of the client software using a Model-View-Controller design paradigm, allowing the graphical view and interface to be developed in a modular fashion. The unique challenge in developing the interface will be that there are multiple separate interfaces which must be included just during the game: one each for each game role. Additionally, a separate layout must be developed for the program in order to allow the user to connect to the server and begin a game. Finally, several special interfaces are required for two infrequent-yet-important game situations: responding to a radar pulse and going on a spacewalk. 

The interface is planned to be implemented in Java's Swing library, as shown in figure \ref{fig:InterfaceModel}. The reason for this choice was primarily due to its familiarity to the members of the team and its simple polymorphic design. The application will display the user one central window capable of taking on alternate arrangements of a core set of display windows. The five primary arrangements, which we refer to as interface modes, correspond to the Captain, First Officer, Engineer, and Radio Officer roles, in addition to a connection settings screen. Contained within these windows should be, in different configurations, sub-panels for the map with the vessel path, map with the speculated enemy path, game announcements, dials representing system status, submarine health, network connection and chat, engineering component status. Each of these panels can be displayed in multiple sizes and in multiple locations on the screen, depending on the interface mode required. 

\newsamimage[0.95]{Sample Component Panels}{gui}

Additional overlay windows will provide the functionality required for Radar and Spacewalk interactions. The radar response window will appear overtop the rest of the GUI of the Captain's client interface when triggered. Other functions of the Captain are disabled, as are the functions of the other members of their team.

\newsamimage[0.5]{Radar Response Overlay}{radar}

When the Captain initiates a spacewalk, each player's standard activities are paused. In a random order, each player must then trace a shape displayed in a pop-up window with their mouse and then click the "next" button, which enables the tracing interface for the next player. After four successful traces around the shapes, the Spacewalk concludes, and the players resume their mission with systems repaired.

\newsamimage[0.5]{Spacewalk Activity}{surface}